La \figurename~\ref{fig:esempio} è una figura di esempio.

\begin{figure}[htbp]
    \centering
    \includegraphics[scale=0.7]{./figures/logo_standard.jpg}
    \caption{Questa è una immagine di esempio}
    \label{fig:esempio}
\end{figure}


Things to add:
\begin{itemize}
    \item Types of test coverage
\end{itemize}



Test case generation can be seen as a multi-objective problem, given that the goal is to cfover multiple test targets.

Search-based approaches for test case generation use optimization algorithms to attempt to find 
the best candidate test case with the objective to maximize fault detection. Genetic Algorithms (GAs) are an example of an
evolutionary search approach for test case generation; starting from an initial, often randomlly genrated, population of 
test cases, the algorithm keeps evolving the individuals according to simulated natural evolution theory principles.
In this context, a typical fitness function of a GA would measure the distance between the execution trace of the generated test cases
and the coverage targets.