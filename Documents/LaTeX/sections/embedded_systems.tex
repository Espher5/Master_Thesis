\section{Overview on Embedded Systems}
\ess can be defined as a combination of hardware components and software systems that seamlessly work together to achieve a specific purpose; they can be dynamically programmed or have a fixed functionality set, and are often engineered to achieve their goal within a larger system. They are commonly found inside devices that we use on a daily basis, such as cell phones, traffic lights, and appliances; here, these systems are responsible for controlling the functions of the device, and they are required to work continuously without the need for human intervention, besides the occasional battery replacement/recharge.
This requirement implies that, in most circumstances, providing maintenance to \ess is challenging or straight up unfeasible; therefore, the design process of a system of this kind must account for a series of additional challenges and constraints, that are typically not considered as much in \noess, in order to ensure their ability to operate in a stand-alone manner and in a wide range of conditions. Many \ess, in fact, are deployed into physical environments that do not have access to a network or are not covered by an internet connection, or are even subject to harsh and adverse weather conditions.
Furthermore, many \ess are used in applications that require a high degree of safety and security, such as in the aerospace or medical industries, meaning that the system must be able to operate without fail and without compromising safety. 

Despite the many challenges, in recent years, \ess have seen a steep surge in popularity, and have driven innovation forward in their respective areas of interest: everywhere, spanning from the agricultural field, to the medical and energy ones, \ess of various size and complexity are employed, especially in areas where human intervention is impractical or straight up impossible.
As the demand for more advanced and sophisticated devices continues to increase, the role of \ess will only become more prominent.



This requires the use of embedded software, which is specifically designed to run on the limited hardware of the system.

There are many different types of \ess, including microcontrollers, digital signal processors (DSPs), and field-programmable gate arrays (FPGAs). Each of these types of systems has its own unique characteristics and is suited to different types of applications.


\section{Enabling technologies}
\subsection{Microcontrollers}
One of the key enabling technologies for \ess is their microcontroller, which is a small single-chip computer that is used to manage the functions of the larger system, in a power-efficient way and ideally at a low cost. While some applications require their own custom-made microcontroller and other custom hardware components built ad-hoc for them, there is a wide variety of general-purpose microcontroller boards, sensors, and actuators that are far easier to program and can also be customized for a high range of applications, which makes them highly versatile.

One type of general-purpose microcontroller that is commonly used for \ess development, as well as for many other IoT purposes, is the Arduino; it is an open-source platform that is based on the Atmel AVR microcontroller; it is widely used in hobbyist and educational projects because of its simplicity and low cost. Many Arduino versions exist, each with a different form factors, amounts of resources on board and I/O pins; as a result they used for applications of increased complexity and needs. Some examples include the Arduino Nano, with the smallest form factor among the Arduino boards and very limited, the Uno, and the Mega.


Another popular general-purpose microcontroller platform is the Raspberry Pi, a small single-board computer that is based on the ARM architecture, which is also what many mobile processors are based on. It comes in different variants, from an Arduino Nano-sized Raspberry Pi Zero and Zero W, but equipped with much more resources, to the Pi 3 and 4 models, which are effectively capable of running a more complex OS, supporting even up to 8GB of RAM.

In addition to these general-purpose microcontrollers, there are also many proprietary devices that are designed for specific applications: given this high specialization, these microcontrollers often offer limited customization capabilities, and may not be easily programmable, if at all, by the user. Some examples of proprietary microcontrollers include the Microchip PIC and the Texas Instruments MSP430; these chips are often used in industrial and commercial applications, where a high level of performance and reliability is required. They may also be used in applications where security is a concern, as their design may be kept confidential to protect against tampering or reverse engineering attempts.

Overall, the choice of microcontroller for an \es will depend on the specific requirements of the user. General-purpose microcontrollers such as Arduino and Raspberry Pi may be suitable for hobbyist or educational projects, while proprietary microcontrollers may be better suited for industrial or commercial applications where performance and reliability are critical.


\subsection{Embedded software and communication protocols}
Software-wise, more complex \ess can be equipped with their own \textbf{Embedded Operating System} (EOS), specifically designed to run on embedded devices, as they have a smaller footprint and fewer features compared to general-purpose OSs like Windows or Linux, and are optimized  for low power consumption. Popular EOSs include TinyOS and Embedded Linux.
Furthermore, the real-time requirements of some \ess calls for their own \textbf{Real-Time Operating System} (RTOS); these are specialized OSs that are designed to provide a predictable response time to events, even when there are many tasks running concurrently. RTOS are essential for \es that require fast and reliable performance, such as in aircraft control systems or medical devices. A notable and open-source example is FreeRTOS. 


At the lower level 
The three main protocols used for low level communications are UART, I2C, and SPI:
\begin{itemize}
    \item \textbf{Universal Asynchronous Receiver/Transmitter (UART)}: one of the simplest protocols used for serial communication between devices; it is full-duplex, which means that data can be transmitted and received at the same time. UART is widely supported by many microcontrollers and microprocessor devices, however, it is typically slower than other communication protocols and can be less reliable over longer distances.
    \item \textbf{Inter-Integrated Circuit (I2C)}: a two-wire communication protocol that is used for communication between devices on the same circuit board; it is a half-duplex protocol, so data can only be transmitted or received at a time. I2C is commonly used to connect devices such as sensors, displays, and memory to a microcontroller. It is relatively fast and can support multiple devices on the same bus; however, it still has a limited data rate.
    \item \textbf{Serial Peripheral Interface (SPI)}: a two-wire communication protocol used for communication between devices; it is a synchronous protocol, which means that data is transmitted and received in a coordinated manner. SPI is relatively fast and can support multiple devices on the same bus, however, it requires more wires and pins compared to I2C and as a reseult can be more complex to implement.
\end{itemize} 



Often, multiple devices are deployed as part of the same larger system and must be able to efficiently communicate between one another to achieve their purpose; therefore, it is essential for \ess to be equipped with a robust suite of protocols, libraries and frameworks.
As always, determining which protocol to employ depends on the constraints the system is dealing with, such as being limited to a low power consumption or being required to maintain a low-latency communication.

Zigbee is a wireless communication protocol specifically designed and built for low-power, low-data-rate applications \cite{Zigbee}; it is often used in sensors and other devices that need to communicate over short distances, such as in home automation systems or industrial control systems. One of its key advantages is its very low power consumption, which makes it so ZigBee is one of the most well-suited protocols for use in devices that need to operate for long periods of time without access to a power source (i.e., a quite substantial subset of \ess).

Bluetooth is another wireless protocol that is commonly used in embedded systems which was designed for medium-range communication and is most commonly used to connect devices such as phones, tablets, and laptops to other devices, such as speakers, keyboards, or headphones. Bluetooth is a widely supported standard and is often used in applications where compatibility with a range of different devices is important; furthermore, with its low-energy variant, Bluetooth can help further optimize power consumption in devices that require it.

SigFox on the other hand, is a low-power, wide-area (LPWA) communication protocol that is designed for Internet of Things (IoT) applications; it is used to transmit small amounts of data over long distances, making it well-suited for use in remote monitoring systems or other applications where conventional communication methods are not practical.

For network communications, IP, and it's low energy version 6LoWPAN, are widely used protocols. It is used to transmit data between devices and is a key component of the internet. IP is often used in embedded systems that need to communicate with other devices or with the internet, such as in smart home systems or industrial control systems.

Overall, each of these protocols has its own strengths and is well-suited for different types of applications. Zigbee is ideal for low-power, low-data-rate applications, while Bluetooth is well-suited for medium-range communication. Sigfox is ideal for long-range communication in IoT applications, and IP is widely used for communication over networks.



\section{Design}
From a design and development point of view, working with \ess can be complex, as it involves a wide range of skills and disciplines, including computer science, electrical engineering, and mechanical engineering. It is often necessary to work closely with other team members, including hardware and software designers, to ensure that the system meets all the requirements.

Failures in \ess should always be evident and identifiable quickly (a heart monitor should not fail quietly \cite{MakingEmbeddedSystems}). Given the high criticality of such systems, ensuring their dependability over the course of their lifespan is essential; \ess can be deployed in extreme conditions (i.e., weather monitoring in extreme locations of the planet, devices inside the human body, or \dots), where maintenance operations cannot be performed regularly,  and high availability is expected. 

The dependability of a system can be expressed in terms of:
\begin{itemize}
    \item \textbf{System maintainability}: the extent to which a system can be adapted/modified to accommodate new change requests. As \ess becomes more complex and feature-rich, it is becoming increasingly important to design them with maintainability in mind. This includes designing systems that are easy to update and repair, as well as ensuring that they can be easily replaced if necessary.
    \item \textbf{System reliability}: the extent to which a system is reliable with respect to the expected behavior.
    \item \textbf{System availability}: the extent to which a system remains available for its users.
    \item \textbf{System security}: the extent to which a system can keep data of its users safe and ensures the safety of their users.
\end{itemize}

These dependability attributes cannot be considered individually, as there are strongly interconnected; for instance, safe system operations depend on the system being available and operating reliably in its lifespan. Furthermore, an \es can be unreliable due to its data being corrupted by an external attack or due to poor implementation. As a result of particular care should be applied in the design of these systems, especially \dots 


In conclusion, \ess are specialized computer systems that are designed to perform a specific task within a larger system; they are able to perform real-time processing and operate in a stand-alone manner, but they also face the challenge of limited resources and power. Despite this, the use of \ess has grown significantly, and they are now found in a wide variety of applications.



\section {Challenges}
An important, and often critical, aspect of ES, which defines the greatest challenges when engineering them, is the limited quantity resources available: these systems often have very small amounts of memory and processing power, so it is important to carefully design the software to make the most efficient use of them. This can involve using specialized programming languages and techniques, such as real-time operating systems and low-level hardware access. Furthermore, many such systems may be powered by using a battery, and thus the hardware they are equipped with, often purpose built, must be highly efficient in its operations. Furthermore, from the software point-of-view, it is essential that the system operates deterministically and with real-time constrains.

Another notable challenge is the requirement of some systems to perform real-time processing tasks. This means that they must be able to process data and provide a response within a specific time frame. For example, an embedded system in a car might be responsible for controlling the engine and transmission; it must be able to process data from sensors and make decisions about how to control the engine and transmission in real-time, as the car is being driven.

In addition to the technical challenges, there are also many non-technical factors that must be considered when developing embedded systems. For example, the system must be able to operate within the physical constraints of the device it is being used in, and it must be able to withstand the environmental conditions in which it will be used.



\section{Testing Embedded Systems}
Testing \ess poses a series of challenges compared to traditional testing: first of all, in the case of ES that are highly integrated with a physical environment (such as with CPSs), replicating the exact conditions in which the hardware will be deployed may be challenging. Secondly, field-testing of these systems can be unfeasible to dangerous or impractical environmental conditions (i.e., a nuclear power plant, a deep-ocean station, or the human body). Furthermore, given the absence of a user interface in most cases, testing such systems can be particularly challenging, given the lack of immediate feedback. Finally, the testing of time-critical systems has to validate the correct timing behavior which means that testing the functional behavior alone is not sufficient; similarly, system with tight hardware constraints, such as memory, limited processor power, or power consumption are difficult to design and test.

Going through multiple hardware revisions in order to meet the requirements can be extremely expensive.

\subsection{X-in-the-loop}
For these reasons and more, the general testing process of \ess follows the X-in-the-loop paradigm \cite{DBLP:journals/software/GarousiFKY18} where the system goes through a series of step that simulate its behavior with an increased level of detail before being actually deployed; subcategories in this area include Model-in-the-Loop, Software-in-the-Loop, Processor-in-the-Loop, Hardware-in-the-Loop, and System-in-the-Loop:
\begin{itemize}
    \item With \textbf{Model-in-the-Loop (MIL)} or \textbf{Model-Based Testing} an initial model of the hardware system is built in a simulated environment; this coarse model captures the most important features of the hardware system \cite{XLoop}. As the next step, the controller module is created, and it is verified that the controller can manage the model, as per the requirements. Commonly, after the testers establish the correct behavior of the controller, its inputs and outputs are recorder, in order to be sued in the later stages of verification.
    \item With \textbf{Software-in-the-Loop (SIL)}, the algorithms that define the controller behavior are implemented in detail, and used to replace the previous controller model; the simulation is then executed with this new implementation. This step will determine whether the control logic, i.e., the Controller model can be actually converted to code and, more importantly, if it is hardware implementable. Here, the inputs and outputs should be logged and matched with those obtained in the previous phase; in case of any substantial differences, it may be necessary to backtrack to the MIL phase and make the necessary changes, before repeating the SIL step. On the other hand, if the performance is acceptable and falls into the acceptance threshold, we can move to the next phase.
    \item The next step is \textbf{Processor-in-the-Loop (PIL)}; here, an embedded processor will be simulated in detail and used to run the controller code in a closed-loop simulation. This help can help determine if the chosen processor is suitable for the controller and can handle the code with its memory and computing constraints. At this point, 
    \item Finally, \textbf{Hardware-in-the-loop} is the last step performed before deploying the \es to the actual hardware. Here, we can run the simulated system on a real-time environment, such as SpeedGoat \cite{SpeedGoat}. The real-time system performs deterministic simulations and has physical connections to the embedded processor, i.e., analog inputs and outpus, and communication interfaces, such as CAN and UDP. This can help identify issues related to the communication channels and I/O interface. HIL can be very expensive to perform and in practice it is used mostly for safety-critical applications, and it is required by automotive and aerospace validation standards. 
\end{itemize}

After these steps, the system can finally be deployed on real hardware.

A common environment for performing the simulation steps discussed above is Simulink \cite{Simulink}; it is a graphical modeling and simulation environment for dynamic systems based on blocks to represent different parts of a system: a block can represent a physical component, a function, or even a small system. Some notable features include: scopes and data visualizations for viewing simulation results, legacy code tool to import C and C++ code into templates and building block libraries for modeling continuous and discrete-time systems.



