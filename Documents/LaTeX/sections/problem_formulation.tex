
\section{Test Driven Development}


\section{Testing Embedded Systems}
Embedded Systems (ES) can be defined as a combination of hardware components and software systems that seamlessly work together to achieve a specific purpose. Such systems can be dynamically programmed or have a fixed functionality set, and are often engineered to achieve a domain-specific, often critical, goal.
In recent years, such system have ... and have driven innovation forward in their respective areas of deployment: everywhere, spanning from the agricultural field, to the medical and energy ones, ES of various size and complexity are employed.


Given the high criticality of such systems, ensuring their dependability over the course of their lifespan is essential; ES can be deployed in extreme conditions and


Due to their high specialization, embedded software often deals with time and resource constraints

Furthermore, given the absence of a user interface in most cases, testing such systems can be particularly challenging, given the lack of immediate feedback.





\section{Test Driven Development for Embedded Systems}
Generally, the testing process of ES follows the X-in-the-loop paradigm; subcategories in this area include model-in-the-loop, software-in-the-loop, processor-in-the-loop, hardware-in-the-loop, and system-in-the-loop.
Reference the old survey papers (i.e. X in the loop) that provide a summary of the main techniques