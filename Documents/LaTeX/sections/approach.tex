In this chapter we will present in detail the planning and approach we followed to establish our study.

the two experimental tasks on which this study is based on, as well as the analysis of the gathered results.

The study was conducted with the participation of 9 undergraduate master's degree students enrolled in the "Embedded Systems" course at the University of Salerno in Italy. The participation voluntarily accepted to take part in this study.


\section{Research Questions}
The study we performed aimed at answering the following Research Questions (RQ):
\begin{itemize}
    \item \textbf{RQ1}:
    \item \textbf{RQ2}: Are there differences between TDD and NO-TDD in the external quality of the implemented solutions, developers’ productivity, and number of tests written?
    \item \textbf{RQ3}: Are there differences between TDD and NO-TDD in (cylcomatic, cognitive and code smells)
\end{itemize}




\section{Experimental tasks}
The participants for this study were students at the University of Salerno, in Italy, they were a mix of second-year master's degree students at the University of Salerno and students visiting with the Erasmus program; this last group was further made up of master's degree student and third-year bachelor's degree students. Both groups were enrolled in the Embedded Systems course at the University of Salerno.

he course covered the following topics, software
quality, unit testing, integration testing, SOLID principles, refactor-
ing, iterative test-last development, big-bang testing, and TDD. The
course included frontal lectures, laboratory sessions, and home-
work. During the laboratory sessions, the students improved their
knowledge about how to develop unit tests in Java by using the
Eclipse IDE and JUnit, and the refactoring functionality available in
Eclipse. During laboratory sessions and by developing homework,
the students practiced unit testing, iterative test-last development,
big-bang testing, and TDD

Participating in this study was voluntary; the students were informed that any gathered data would be treated anonymously and shared for research purposes only. Furthermore, participation would not directly affect their final mark for the Embedded System course; however, in order to encourage student participation, those who took part in the study were rewarded with a bonus in their final mark. Among the students taking the course, 9 participated in this study.

The participants for the tasks were asked to carry out their assignment by either using TDD or no-TDD (i.e., any approach they preferred, except for TDD) depending on the group they were partitioned in. and on the period the task took place.


\section{Variables}
Independent variables:
\begin{itemize}
    \item \textbf{Technique}: a nominal variable assuming two values, TDD and no-TDD.
    \item \textbf{Period}: since the study is longitudinal - i.e., the data was collected over time - we took into account the period during which each treatment (TDD or no-TDD) was applied.
    \item \textbf{Group}:  variable representing the two groups. It can assume the values G1 and G2.
\end{itemize}


Dependent variables:
\begin{itemize}
    \item \textbf{Quality (QLTY)}: quantifies the external quality of the solution a participant implemented. This variable is defined as follows: ...
    \item \textbf{Productivity (PROD)}: estimates the productivity of a participant. It is computed as follows: ...
    \item \textbf{Number of tests (TEST)}: quantifies the number of unit tests a participant wrote.
\end{itemize}


Furthermore, we considered three additional dependent variables regarding the code smells in the submitted projects were collected:
\begin{itemize}
    \item \textbf{Number of smells (SMELL)}
    \item \textbf{Cyclomatic Complexity (CYC)}
    \item \textbf{Cognitive Complexity (COG)}
\end{itemize}




\section{Study design}
The participants were randomly split into two groups, G1 and G2, having 4 and 5 participants respectively. For the first period P1, the group G1 was assigned the TDD task, while the group G2 was assigned the no-TDD task; on the other hand, during period P2, the group G1 was assigned the no-TDD task, while the group G2 was assigned the TDD task.
Therefore, the design of our study can be classified as a repeated-measures, within-subjects design. In each period, the participants in G1 and G2 dealt with different experimental objects. For instance, in P1, the participants in G1 dealt with BSK, while those in G2 with MRA. At the end of the study, every participant had tackled each experimental object only once.

\section{Procedure}
Before our study took place, we collected some demographic information on the participants. To this end, the participants were asked to fill out an on-line pre-questionnaire (created by means of Google Forms).

The Embedded Systems course, during which the study was conducted, started in September 2022. The first task, P1, took place on Tuesday, December 6th 2022, while the second, P2, took place on Tuesday, December 13th 2022.

Between the start of the course and P1, the participants had never dealt with TDD, while they were somewhat familiar with unit testing and iterative test-last development. In the weeks before P1, TDD was introduced to the participants via some lectures; also had taken part in ... training sessions on TDD and completed some homework by using this development practice.


\section{Results}