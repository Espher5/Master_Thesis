In this chapter we will present in detail the approach we followed to establish the two experimental tasks on which this study is based on, as well as the analysis of the gathered results.

The study was conducted with the participation of 9 undergraduate master's degree students enrolled in the "Embedded Systems" course at the University of Salerno in Italy. The participation voluntarily accepted to take part in this study.





\section{Research Questions}
The study we performed aimed at answering the following Research Questions (RQ):
\begin{itemize}
    \item \textbf{RQ1}:
    \item \textbf{RQ2}:
    \item \textbf{RQ3}:
\end{itemize}




\section{Experimental tasks}
The following table contains a summary of the user stories for the two experimental tasks on which the study was conducted:


The participants for the tasks were asked to carry out their assignment by either using TDD or no-TDD (i.e., any approach they preferred, except for TDD) depending on the group they were partitioned in. and on the period the task took place.

Independent variables:
\begin{itemize}
    \item \textbf{Technique}: a nominal variable assuming two values, TDD and no-TDD.
    \item \textbf{Period}: since the study is longitudinal - i.e., the data was collected over time - we took into account the period during which each treatment (TDD or no-TDD) was applied.
    \item \textbf{Group}:  variable representing the two groups. It can assume the values G1 and G2.
\end{itemize}


Dependent variables:
\begin{itemize}
    \item \textbf{Quality (QLTY)}: quantifies the external quality of the solution a participant implemented. This variable is defined as follows: ...
    \item \textbf{Productivity (PROD)}: estimates the productivity of a participant. It is computed as follows: ...
    \item \textbf{Number of tests (TEST)}: quantifies the number of unit tests a participant wrote.
\end{itemize}


Furthermore, we considered three additional dependent variables regarding the code smells in the submitted projects were collected:
\begin{itemize}
    \item \textbf{Number of smells (SMELL)}
    \item \textbf{Cyclomatic Complexity (CYC)}
    \item \textbf{Cognitive Complexity (COG)}
\end{itemize}




\section{Study design}
The participants were randomly split into two groups, G1 and G2, having 4 and 5participants respectively. For the first period P1, the group G1 was assigned the TDD task, while the group G2 was assigned the no-TDD task; on the other hand, during period P2, the group G1 was assigned the no-TDD task, while the group G2 was assigned the TDD task.
Therefore, the design of our study can be classified as a repeated-measures, within-subjects design. In each period, the participants in G1 and G2 dealt with different experimental objects. For instance, in P1, the participants in G1 dealt with BSK, while those in G2 with MRA. At the end of the study, every participant had tackled each experimental object only once.

\section{Procedure}
Before our study took place, we collected some demographic information on the participants. To this end, the participants were asked to fill out an on-line pre-questionnaire (created by means of Google Forms).

The Embedded Systems course, during which the study was conducted, started in September 2022. The first task, P1, took place on Tuesday December 6th 2022, while the second, P2, took place on Tuesday December 13th 2022.

Between the start of the course and P1, the participants had never dealt with TDD, while they were somewhat familiar with unit testing and iterative test-last development. In the weeks before P1, TDD was introduced to the participants via some lectures; also had taken part in ... training sessions on TDD and completed some homework by using this development practice.


\section{Results}