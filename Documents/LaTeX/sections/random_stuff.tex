Testing approaches for embedded systems:
One of the most popular approaches is model-based testing. A model is generated either with forward or backward engineering
	- With forward engineering, the model is designed after the requirement analysis phase.
	- With backward engineering, the model is retrieved by performing reverse engineering on the system's implementation

Models can then be used for test-case generation; often FSMs are used for this step with coverage criteria such as all-transition coverage. Other approaches include MATLAB Simulink models, Model Definition Language, and UML sequence diagrams.

Model-based testing can be generalized into the X-in-the-loop paradigm, which consists of Model-in-the-Loop (MiL), Software-in-the-Loop (SiL), Processor-in-the-Loop (PiL), Hardware-in-the-loop (HiL), and System-in-the-Loop (SYSiL).




Diversity promoted in both AmbieGen and DeepJanus

To produce a realistic road, DeepJanus uses the Catmull-Rom cubic splines approach, interpolating its curves to obtain a 2D point sequence

Catmull Splines ref: Edwin Catmull and Raphael Rom. 1974. A Class of Local Interpolating Splines.
In Computer Aided Geometric Design, R. E. Barnhill and R. F. Riesenfeld (Eds.).
Academic Press, 317 – 326. https://doi.org/10.1016/B978-0-12-079050-0.50020-5