\chapter*{Abstract}
In this thesis, we define two experiments (\ie a baseline experiment and its replication) and present their results; the goal of these experiments is to compare Test-Driven Development (\tdd), an incremental approach to software development where tests are written before production code, with a traditional way of testing where production code is written before tests (\ie \notdd). 
\tdd has been the subject of numerous studies over the years, with the purpose of determining whether applying this technique would result in an improved development: in this study, \tdd and \notdd have been contrasted in the context of the implementation of Embedded Systems (\ess).
During the baseline experiment we provided the participants with two small \ess to implement and test in a host development environment, while mocking the underlying hardware platform.
As for the replication study, its main goal is to validate the results obtained from the baseline experiment and to generalize them to a different and more real setting. 
In particular, during the replication experiment, we asked participants to implement another \es, which this time had to be effectively deployed and tested on a hardware platform (\ie a \textit{Rapberry Pi} model 4); this system was also tested before the deployment.

Given the small number of participants to this study, we consider this assessment the first exploratory step to the research on the topic of \tdd for \ess, given the fact that there are no similar studies available yet. As a result, the gathered data cannot prove a statistically significant difference between the two approaches; however, this data provided interesting cues to determine where to head with a future, more extensive, research on the topic.

From the two experiment we gathered quantitative data; moreover, qualitative data was gathered to explain the quantitative measurements obtained, and to obtain a better understanding of the phenomenon under study.
After the analysis, data suggests that developers' productivity and number of written test cases increase when using TDD to develop \ess, while there is not a substantial difference with respect to the external quality of the developed implementation. 
Finally, developers' perspectives highlight how \tdd is perceived as a more difficult approach to apply compared with \notdd.

\ \\ \
\noindent \textbf{Keywords}: Embedded Systems, Empirical Assessment, Experiment, Software Testing, Test-Driven Development.