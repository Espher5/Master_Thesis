In this thesis, we define two experiments (\ie, a baseline experiment and its replication) and analyze their results. The goal of these experiments is to compare Test-Driven Development (\tdd), an incremental approach to software development where tests are written before production code, with a traditional way of coding where production code is written before tests (\ie, \notdd). \tdd has been the subject of numerous studies over the years, with the purpose of determining whether applying this technique would result in an improved development. In this study, \tdd and \notdd have been contrasted in the context of the implementation of embedded systems (\ess). The goals of the replication study are to validate the results of the baseline experiment and to generalize them to a different and more real setting. The most remarkable differences concern the implementation task and the experimental procedure: as for the task, we asked the participants to implement a larger and more complex \es in the replicated experiment; as for the procedure, the implementation task was not accomplished in the replicated experiment under controlled conditions and the developed \es (\ie, the result of such task) was then deployed and tested on a real software/hardware environment.  
The Participants in the experiments were final year students in Computer Science enrolled to the \textit{Embedded Systems} course at the University of Salerno, in Italy. Before taking part in the experiment, they were trained on concepts spanning from unit testing and its guidelines, to the introduction of \tdd.
The results obtained by an aggregated analysis of data gathered from both the baseline and the replicated experiments suggest that there is not a significant difference between \tdd and \notdd on: productivity, number of written tests, a... . On the other hand we observed a significant on: ... When analyzing experiments individually, we observed that ... 
\\ \ \\
\noindent \textbf{Keywords}: Embedded systems, empirical assessment, software testing, test-driven development.