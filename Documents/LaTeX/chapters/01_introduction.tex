
A computer hardware and software combination created for a particular purpose is an Embedded System (\es). In many cases, \ess operate as part of a bigger system (\eg  agricultural and processing sector equipment, automobiles, medical equipment, airplanes, and so on) and within tight resource constrains (\eg small battery capacity, limited memory and CPU speed, and so on).
The global \ess market is expected to witness notable growth. A recent report evaluated the global \ess market 89.1 billion dollars in 2021, and this market is projected to reach 163.2 billion dollars by 2031; with a compound annual growth rate of 6.5\%~ \cite{ESSTR2022}. This growth is mostly related to an increase in the demand for advanced driver-assistance system (in electric and hybrid vehicles) and in the number of \ess-related research and development projects.  
To date, there has yet to be shown which approach is to be preferred when developing \ess. For example, Greening \cite{TDDEC} in his book asserted that embedded developers can benefit from the application of
Test-Driven Development (\tdd), an incremental approach to software development in which a developer repeats a short development cycle made up of three phases: \textit{red}, \textit{green}, and \textit{blue/refactor} \cite{TDDByExample}. During the red phase, the developer writes a unit test for a small chunk of a functionality not yet implemented and watches the test fail. In the green phase, the developer implements the chunk of functionalities as quickly as possible and watches all the unit tests pass. During the refactoring phase, the developer changes the internal structure of the code while paying attention to keep the functionality intact—accordingly, all unit tests should pass. TDD has been conceived to develop “regular” software, and it is claimed to improve software quality as well as developers' productivity \cite{DBLP:reference/se/ErdogmusMJ10}. \ess have all the same challenges of non-embedded systems (\noess), such as poor quality, but add challenges of their own [6]. 
For example, one of the most cited differences between embedded and \noess is that embedded code depends on the hardware. While in principle, there is no difference between a dependency on a hardware device and one on a \noess \cite{TDDEC}, dealing with hardware introduces a whole new set of variables to consider during development. Furthermore, the limited resources on which an \es usually operates may make it extremely difficult to properly test the system in its entirety. 
Over the years, a huge amount of empirical investigations has been conducted to study the claimed effects of TDD on the development of \noess (\eg, \cite{DBLP:journals/software/KaracT18}). So far no investigations have been conducted to assess possible benefits concerned the application of TDD on the development of \ess although some authors, like Greening \cite{TDDEC}, believes that embedded developers can benefit from the application of TDD in the development of \ess.

In this work of thesis, we investigate the following primary research question (RQ):
\\ \ \\
\noindent \textbf{RQ.} To what extent does the use of \tdd impact the external quality and productivity of the developed \es?	
\\ \ \\
To answer this RQ, we present the results of an exploratory empirical assessment constituted of two experiment, a controlled experiment that acts as the baseline for the study, and its replication to study the application of \tdd on the implementation of \ess. The participants were final year Master's degree students in Computer Science enrolled to an \es course at the University of Salerno, in Italy. The goal of these experiments is to increase the body of knowledge on the benefit (if any) related to the application of \tdd in the context of the development of \ess. To that end, we compare \tdd with respect to a more traditional, test-last, development practice, where test cases are written after the production code. From here onwards, we refer to this traditional way of coding as \notdd. Whichever the used approach (\tdd and \notdd), the participants in the implementation of an \es where asked to use a mock to model and confirm the interactions between a device driver and the hardware. The mocked implementation would intercept commands to and from the device simulating a given usage scenario. In the replication experiment, the participants had to implement an additional \es with the end goal of replacing the mocks with real hardware components (a number of sensors and actuators) before deploying the \es on the actual hardware platform they had mocked up to that point. The logic of the \es was deployed on a Raspberry Pi model 4 board, and the developed test cases were executed in the real environment. Variations in the replicated experiments (task and the experimental procedure) were introduced to validate the results of the baseline experiment and to generalize these results to a more real setting. The results obtained by an aggregated analysis of the data from both the experiments suggest that there are no significant differences between \tdd and \notdd on: productivity, number of written tests, a..., respectively. On the other hand, we observed a significant difference on: ... When analyzing experiments individually, we observed that ...

As for the following thesis structure, it is made up of five chapters:
The first two act as the foundational knowledge concepts that provide a general overview on the main topics concerning the \tdd methodology and \es technologies: chapter one contains an overview on the software testing process, analyzing the main approaches, with a focus on \tdd. Chapter two will provide information on the general \ess concepts, enabling technologies and implementation challenges, before discussing the techniques for testing such systems.
In chapter three, we conduct a review of the literature by examining the most relevant previous empirical studies on the application of \tdd and the current testing methodologies for \es.
Chapter four will contain the detail explanation of our approach for the definition of the two experimental studies, the analysis of the results, and our answers to the research questions.
Finally, chapter five will provide the conclusions to the thesis, along with a discussion on the possible ramification the research could manifest when moving forward in the analysis of \tdd for \es development.