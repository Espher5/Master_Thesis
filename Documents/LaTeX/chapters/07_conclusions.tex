\chapter{Conclusions and final remarks}
\label{chap:7_conclusions}
In this thesis we presented an empirical assessment constituted of two experiments, a baseline (\textit{Exp1}) and its replication (\textit{Exp2}), with the objective of investigating how \tdd impacts the development of \ess and how it compares to traditional test-last approaches, especially in terms of the external quality of the implemented \es, developers' productivity, and number of written test cases. 
Participants to this study accomplished three implementation tasks in \textit{Python} and targeting the \textit{Raspberry Pi model 4} hardware platform, while using either \tdd or \notdd to test their developed \ess.

For the first and second tasks, participants made use of mock objects to model the interaction with the underlying hardware components (according to the pipeline proposed by Greening \cite{TDDEC}), while developing and testing their implementation in a host development environment. As for the final tasks, around which the replication study was designed, participants used the same approach, before deploying and testing their solution on the real hardware.

The overall results of the experiments suggest that external quality of the developed \ess and number of written test cases increase when applying \tdd, while there is not a substantial difference with respect to the developers'productivity. Furthermore, participants expressed how learning \tdd was harder, especially those who were strongly used to a test-last approach; we can see this reflected in the implementation of the second experimental task, which saw on average better performances for \notdd. However, some members of the \tdd group, which we speculate had a stronger reception of this approach during the training sessions, still performed better than the \notdd group.

Based on the gathered data, we delineated possible implications from the perspectives of lecturers and researchers. For example, our results suggest lecturers teach \tdd along with a development pipeline where hardware components are mocked before actually deploying the \es in the environment for which it has been developed. 

Finally, despite we gathered evidence that \tdd can be successfully applied to the development of \ess, we foster replications of our experiments, especially by involving professional developers and the software industry. 
Our investigation has the merit to justify such replications, as it is easier to recruit professional developers when initial evidence is available. 

\subsection{Individual contribution to the experimentation}
This small section concerns the individual contribution that I, as the author of this thesis, provided towards the planning, execution, and analysis of this study.
Specifically, the steps I performed - overseen by my supervisors - are:

\begin{itemize}
    \item \textbf{Participation to the planning of the study}. As the first step I discussed, along with my supervisors, interesting cues on which to focus for an experimental study on \tdd applied to \es.  

    \item \textbf{Development of the experimental material}. I assembled  the teaching material that was used for the initial lectures and training/homework sessions that took place before the beginning of the experiment. This material was made up of presentations, exercises, and project templates.
    Similarly, I designed the experimental tasks, by defining a document for each of them, containing the goal of the task, development instructions, and the list of user stories to implement. Furthermore, for each task, I created a GitHub repository containing the project template that was later shared with the participants.
    Finally, for the first two tasks, I defined the post-questionnaires that were filled by the students after they completed the implementations, as well as the script for the final interview, to be performed at the end of the study.

    \item \textbf{Execution of the studies}. 
    I held the lectures and interactive training sessions to provide the students with the knowledge necessary to tackle the experiments. 
    Moreover, I oversaw the students during the execution of the first two experimental tasks, and provided assistance and clarification when necessary.

    \item \textbf{Hardware deployment for the replication experiment}.
    \dots
    
    \item\textbf{Final interviews}.
    Once the hardware implementation was tested, I sat down with each participant individually and interviewed them according to the prepared template, in order to gather their feedback on the whole study, from the lectures, to the last task they implemented.

    \item \textbf{Data extraction}. 
    Besides designing the experimental tasks, I also defined the respective acceptance test suite for each of them; these suites were organized in such a way that their execution was automated, and the test outputs were logged.
    Once the participants to the study had submitted their developed system, I implemented a program to automatically scan their projects and run the acceptance suites on their source code, with the goal of extracting the metrics on which the analysis was later performed. 
    After each execution, the \dots

    \item\textbf{Data analysis}.
    After extracting the raw data and organizing it in a spreadsheet, I built a notebook in which I performed  \dots
\end{itemize}