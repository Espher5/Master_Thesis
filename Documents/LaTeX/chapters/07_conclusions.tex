\chapter{Conclusions}
\label{chap:7_conclusions}
To conclude, in our research for this work of thesis we presented an empirical assessment constituted of two experiments, a baseline (\textit{Exp1}) and its replication (\textit{Exp2}), with the objective of investigating how \tdd impacts the development of \ess and how it compares to traditional test-last approaches, especially in terms of the external quality of the implemented \es, developers' productivity, and number of written test cases. 
Participants to this study accomplished three implementation tasks in \textit{Python} and targeting the \textit{Raspberry Pi model 4} hardware platform, while using either \tdd or \notdd to test their developed \ess.

The overall results of the experiments suggest that developers' productivity and number of written test cases increase when applying \tdd, while there is not a substantial difference with respect to the external quality of the developed \ess.
Based on the gathered data, we delineated possible implications from the perspectives of lecturers. For example, our results suggest lecturers teach \tdd along with a development pipeline where hardware components are mocked before actually deploying the \es in the environment for which it has been developed. It is also worth mentioning that developers perceived \tdd as more difficult than \notdd; this opens a number of research paths to better understand this phenomenon.

Finally, despite we gather evidence that \tdd can be successfully applied to the development of \ess, we foster replications of our experiments, especially by involving professional developers and the software industry. 
Our investigation has the merit to justify such replications — it is easier to recruit professional developers when initial evidence is available. 