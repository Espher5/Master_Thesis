\chapter{Conclusions}
To conclude, in our research for this work of thesis we presented an empirical assessment constituted of two experiments, a baseline and its replication, with the objective of investigating how \tdd impacts the development of \ess and how it compares to traditional test-last approaches, especially in terms of the external quality of the implemented \es, developers' productivity, and number of written test cases.

Participants to this study ... in \textit{Python} and targeting the \textit{Raspberry Pi} hardware platform, by mocking its interfaces in software.


Finally, despite we gather evidence that \tdd can be successfully applied to the development of \ess, we foster replications of our experiments, especially by involving professional developers and the software industry. Our exploratory investigation has the merit to justify such replications — it is easier to recruit professional developers when initial evidence is available. To ease the replicability of our investigation, our online material includes a replication package.