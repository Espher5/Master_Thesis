\chapter{Conclusions and final remarks}
\label{chap:7_conclusions}
To conclude, in our research for this work of thesis we presented an empirical assessment constituted of two experiments, a baseline (\textit{Exp1}) and its replication (\textit{Exp2}), with the objective of investigating how \tdd impacts the development of \ess and how it compares to traditional test-last approaches, especially in terms of the external quality of the implemented \es, developers' productivity, and number of written test cases. 
Participants to this study accomplished three implementation tasks in \textit{Python} and targeting the \textit{Raspberry Pi model 4} hardware platform, while using either \tdd or \notdd to test their developed \ess.

For the first and second tasks, participants made use of mock objects to model the interaction with the underlying hardware components (according to the pipeline proposed by Greening \cite{TDDEC}), while developing and testing their implementation in a host development environment. As for the final tasks, around which the replication study was designed, participants used the same approach, before deploying and testing their solution on the real hardware.

The overall results of the experiments suggest that external quality of the developed \ess and number of written test cases increase when applying \tdd, while there is not a substantial difference with respect to the developers'productivity. Furthermore, participants expressed how learning \tdd was harder, especially those who were strongly used to a test-last approach; we can see this reflected in the implementation of the second experimental task, which saw on average better performances for \notdd. However, some members of the \tdd group, which we speculate had a stronger reception of this approach during the training sessions, still performed better than the \notdd group.

Based on the gathered data, we delineated possible implications from the perspectives of lecturers and researchers. For example, our results suggest lecturers teach \tdd along with a development pipeline where hardware components are mocked before actually deploying the \es in the environment for which it has been developed. 

Finally, despite we gathered evidence that \tdd can be successfully applied to the development of \ess, we foster replications of our experiments, especially by involving professional developers and the software industry. 
Our investigation has the merit to justify such replications, as it is easier to recruit professional developers when initial evidence is available. 
