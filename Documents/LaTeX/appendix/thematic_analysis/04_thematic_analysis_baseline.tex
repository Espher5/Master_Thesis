\chapter{Thematic Analysis - Baseline study}

\section{Template}
\begin{enumerate}
    \item \textbf{Feelings on the development task}:
    \begin{itemize}
        \item Task 1:
        \begin{itemize}
            \item Easy / went smoothly. (\textbf{6})
            \item Some difficulties / needed more time. (\textbf{2})
        \end{itemize}

        \item Task 2:
        \begin{itemize}
            \item No problem with the task. (\textbf{2})
            \item Harder task, found some difficulties. (\textbf{3})
            \item Harder task, but no problem. (\textbf{3})
        \end{itemize}
    \end{itemize}

    \item \textbf{Feelings on \tdd to accomplish the task}:
    \begin{itemize}
        \item Task 1:
        \begin{itemize}
            \item Good experience. (\textbf{1})
            \item Still not sure. (\textbf{2})
        \end{itemize}

        \item Task 2:
        \begin{itemize}
            \item Good experience. (\textbf{1})
            \item Still not sure. (\textbf{1})
            \item Having troubles. (\textbf{3})
        \end{itemize}
    \end{itemize}
    
    \ \\ \
    \item \textbf{Feelings on \notdd to accomplish the task}:
    \begin{itemize}
        \item Task 1:
        \begin{itemize}
            \item Good experience. (\textbf{2})
            \item Concerns on not testing enough some components. (\textbf{1})
            \item \textbf{\notdd approach}
            \begin{itemize}
                \item Wrote tests after the production code. (\textbf{3})
                \item No tests written. (\textbf{2})
            \end{itemize}
        \end{itemize}

        \item Task 2:
        \begin{itemize}
            \item Good experience. (\textbf{2})
            \item Not using \tdd was harder. (\textbf{3})
            \item \textbf{\notdd approach}
            \begin{itemize}
                \item Wrote tests after the production code. (\textbf{4})
            \end{itemize}
        \end{itemize}
    \end{itemize}
    
    \item \textbf{Comparison between \tdd and \notdd}:
    \begin{itemize}
        \item \tdd preference. (\textbf{4})
        \item \notdd preference. (\textbf{3})
        \item Depends on the task. (\textbf{1})
    \end{itemize}
\end{enumerate}

\section{Answers to the questionnaires}
\subsection{Task 1 - Intelligent Office}
\begin{enumerate}
    \item \textbf{Student\_01 (\tdd)}: 
    \begin{itemize}
        \item \textbf{Q1.} NO ANSWER.
        \item \textbf{Q2.} NO ANSWER.
    \end{itemize}

    \item \textbf{Student\_02 (\tdd)}: 
    \begin{itemize}
        \item \textbf{Q1.} I think TDD is very helpful, and I am glad that I can learn it.
        \item \textbf{Q2.} I just needed more time and more exercise.
    \end{itemize}

    \item \textbf{Student\_03 (\tdd)}: 
    \begin{itemize}
        \item \textbf{Q1.} I honestly think \tdd is not as error-prone as regular development. It increases software quality, and it assures that tests exist. But if I'm honest, \tdd is sometimes hard, because you have to think different compared to the usual methods.
        \item \textbf{Q2.} Task 4 - I really had to think deeply into this, because implementing this story broke the tests for the third story.
    \end{itemize}

    \item \textbf{Student\_04 (\tdd)}: 
    \begin{itemize}
        \item \textbf{Q1.} I still have to fully understand how it works.
        \item \textbf{Q2.} I found the task very clear for understanding the various steps to be done.
    \end{itemize}

    \item \textbf{Student\_05 (\notdd)}: 
    \begin{itemize}
        \item \textbf{Q1.} I think that for me it's easier to program a \notdd software, because I never did \tdd before. So, this is my "normal" way to code, because we didn't need to write a test file it was quicker and less work.
        \item \textbf{Q2.} I liked that there were similarities with the previous tasks.
        \item \textbf{Q3 (\notdd testing approach).} I just programmed one task after the other, like we learned in the previous lessons. 
    \end{itemize}

    \item \textbf{Student\_06 (\notdd)}: 
    \begin{itemize}
        \item \textbf{Q1.} For me it is simpler because I can focus on the code.
        \item \textbf{Q2.} User stories were clear and not difficult to implement.
        \item \textbf{Q3.} I still have some difficulties during the test phase, maybe is simpler doing it with TDD approach.
    \end{itemize}

    \item \textbf{Student\_07 (\notdd)}: 
    \begin{itemize}
        \item \textbf{Q1.} \notdd may be faster when you are doing easy tasks but following this kind of approach may incur in some difficult debugging session, since there are possibilities to implement some code that may be never tested or not tested well.
        \item \textbf{Q2.} This development task has been useful to understand the difference with using \tdd and its advantages in respect to use \notdd
        \item \textbf{Q3.} Long story short, I've implemented all the function following the user story and only after I've finished all of them I started writing tests.
    \end{itemize}

    \item \textbf{Student\_08 (\notdd)}: 
    \begin{itemize}
        \item \textbf{Q1.} I'm not used to Python, so the syntax was a bit complicated for me, but \notdd is the classical way of coding and testing for me, so I felt familiar with it.
        \item \textbf{Q2.} Easy to understand, very well-structured, and I knew exactly what to do.
        \item \textbf{Q3.} The knowledge from the previous lectures helped me to accomplish this task.
    \end{itemize}

    \item \textbf{Student\_09 (\notdd)}: 
    \begin{itemize}
        \item \textbf{Q1.} This is a nice activity, but maybe we have to discuss it more, because it is not really clear and there are some unknown in my mind.
        \item \textbf{Q2.} These activities help and improves us.
        \item \textbf{Q3.} NO ANSWER.
    \end{itemize}
\end{enumerate}


\subsection{Task 2 - Cleaning Robot}
\begin{enumerate}
    \item \textbf{Student\_01 (\notdd)}:
    \begin{itemize}
        \item \textbf{Q1.} It was ok.
        \item \textbf{Q2.} It was not so hard because based on the exercises we have done before. 
        \item \textbf{Q3 (\tdd and \notdd comparison).} \tdd is the best one to avoid any errors in the future.
        \item \textbf{Q4 (\notdd testing approach).} I have coded first and then wrote the test.  
    \end{itemize}

    \item \textbf{Student\_02 (\notdd)}:
    \begin{itemize}
        \item \textbf{Q1.} It was hard not to use \tdd.
        \item \textbf{Q2.} It was okay.
        \item \textbf{Q3 (\tdd and \notdd comparison).} I prefer using \tdd.
        \item \textbf{Q4 (\notdd testing approach).} No certain approach.  
    \end{itemize}

    \item \textbf{Student\_03 (\notdd)}:
    \begin{itemize}
        \item \textbf{Q1.} With \tdd I wouldn't have fallen into some pitfalls; I had to rewrite some code because I didn't know I broke some components with future modifications. With TDD I would have known way earlier. 
        \item \textbf{Q2.} It was a brainful task, but still excellent. Wouldn't it be funny to run the code on an actual Roomba?
        \item \textbf{Q3 (\tdd and \notdd comparison).} \tdd: it reduces bugs, if you break something with a refactoring you know it instantly, and the tests are like documentation. \notdd: bugs and functional errors are only known at the end of the testing phase, which for me is terrible. I would now prefer \tdd, but only if the functional requirements are clear; for discovering technology, I wouldn't use \tdd for sure. But I guess it makes sense that you can't write good tests before the implementation for something like a quick throw away prototype.
        \item \textbf{Q4 (\notdd testing approach).} Write all the production code, and then I tested it.
    \end{itemize}

    \item \textbf{Student\_04 (\notdd)}:
    \begin{itemize}
        \item \textbf{Q1.} I feel quite confident.
        \item \textbf{Q2.} I feel quite confident.
        \item \textbf{Q3 (\tdd and \notdd comparison).} With the \notdd approach, I seem to be able to better test the functions I implement. 
        \item \textbf{Q4 (\notdd testing approach).} I implemented all the functions first and then wrote the various tests. 
    \end{itemize}

    \item \textbf{Student\_05 (\tdd)}:
    \begin{itemize}
        \item \textbf{Q1.} During the programming lessons of \tdd, I could only listen to the explanation or try to write down the code from the projector, since for me it was impossible to do both at the same time; I tried to copy the code, but I don't know how to program well the \tdd style. So today it was really hard for me.
        \item \textbf{Q2.} I think the CleaningRobot was harder than the IntelligentOffice, but maybe only because of \tdd. Also, when you write \tdd, you have to write two codes, the actual code and then the whole test code, so it takes more time.
        \item \textbf{Q3 (\tdd and \notdd comparison).} I think that I have made a few mistakes in the \notdd code, but still I got everything, and the mistakes should be easy to fix. So, I prefer \notdd, but maybe also because I'm used to it.
    \end{itemize}

    \item \textbf{Student\_06 (\tdd)}:
    \begin{itemize}
        \item \textbf{Q1.} For me it's more difficult to use this methodology; I prefer to create the code before testing it   
        \item \textbf{Q2.} I had some difficulties to understand the task, it was more complex. 
        \item \textbf{Q3 (\tdd and \notdd comparison).} \notdd was simpler to apply, also the tasks were easier to implement. Anyway, I prefer the \notdd approach.
    \end{itemize}

    \item \textbf{Student\_07 (\tdd)}:
    \begin{itemize}
        \item \textbf{Q1.} At first \tdd may seem trickier, but in a few exercises it became more and more easy to apply.
        \item \textbf{Q2.} The exercise seems a little harder than the last one ,and it required me more time.
        \item \textbf{Q3 (\tdd and \notdd comparison).} Applying \tdd forces me to think in a way I'm not used to: the idea to implement only the code necessary to pass the code it's something I'm not into yet, but I still prefer \tdd in respect to \notdd.
    \end{itemize}

    \item \textbf{Student\_08 (\tdd)}:
    \begin{itemize}
        \item \textbf{Q1.} It is hard to think the other way around as a software developer, but I think in my opinion \tdd is very useful and if you are used to it, it really helps you a lot to improve your programming.
        \item \textbf{Q2.} NO ANSWER.
        \item \textbf{Q3 (\tdd and \notdd comparison).} As I said, the thinking is upside down, and it's kind of like learning a new logical thinking. I still prefer the \notdd because I like it more to write the logic first.
    \end{itemize}

    \item \textbf{Student\_09 (\tdd)}:
    \begin{itemize}
        \item \textbf{Q1.} I think I didn't get it, but I tried so hard.
        \item \textbf{Q2.} If I am still alive I am probably developing my skills.
        \item \textbf{Q3 (\tdd and \notdd comparison).} NO ANSWER,
    \end{itemize}
\end{enumerate}
