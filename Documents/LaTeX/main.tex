\documentclass[11pt,oneside]{book}

\usepackage{array}
\usepackage[printonlyused, withpage]{acronym}
\usepackage[ruled, lined, linesnumbered, noend ]{algorithm2e}
\usepackage{amsmath, amssymb, amsthm}
\usepackage{caption}
\usepackage{color}
\usepackage[english]{babel} 
\usepackage{epsfig}
\usepackage{fancybox}
\usepackage[acronym, nomain]{glossaries}
\usepackage{graphicx}
\usepackage{listings}
\usepackage{microtype}
\usepackage{moreverb}
\usepackage{subfigure}
\usepackage{todonotes}
\usepackage{url}

\usepackage[utf8]{inputenc}
\usepackage[final, bookmarks, breaklinks, colorlinks, allcolors=black]{hyperref}
\usepackage[autostyle=true]{csquotes} % Required to generate language-dependent quotes in the bibliography
\usepackage[sorting=none]{biblatex} % Use the bibtex backend with the authoryear citation style (which resembles APA)
\addbibresource{sections/bibliography.bib} % The filename of the bibliography

\newtheorem{problem}{Problem}

\theoremstyle{definition}
\newtheorem{definition}{Definition}

\definecolor{mygreen}{rgb}{0,0.6,0}
\definecolor{mygray}{rgb}{0.5,0.5,0.5}
\definecolor{mymauve}{rgb}{0.58,0,0.82}
\lstset{ %
	backgroundcolor=\color{white},   % choose the background color; you must add \usepackage{color} or \usepackage{xcolor}; should come as last argument
	basicstyle=\footnotesize,        % the size of the fonts that are used for the code
	breakatwhitespace=false,         % sets if automatic breaks should only happen at whitespace
	breaklines=true,                 % sets automatic line breaking
	captionpos=b,                    % sets the caption-position to bottom
	commentstyle=\color{mygreen},    % comment style
	deletekeywords={...},            % if you want to delete keywords from the given language
	escapeinside={\%*}{*)},          % if you want to add LaTeX within your code
	extendedchars=true,              % lets you use non-ASCII characters; for 8-bits encodings only, does not work with UTF-8
	frame=single,
	keepspaces=true,                 % keeps spaces in text, useful for keeping indentation of code (possibly needs columns=flexible)
	keywordstyle=\color{blue},       % keyword style
	morekeywords={*,...},            % if you want to add more keywords to the set
	numbers=left,                    % where to put the line-numbers; possible values are (none, left, right)
	numberstyle=\tiny\color{mygray}, % the style that is used for the line-numbers
	rulecolor=\color{black},         % if not set, the frame-color may be changed on line-breaks within not-black text (e.g. comments (green here))
	showspaces=false,                % show spaces everywhere adding particular underscores; it overrides 'showstringspaces'
	showstringspaces=false,          % underline spaces within strings only
	showtabs=false,                  % show tabs within strings adding particular underscores
	stringstyle=\color{mymauve},     % string literal style
	tabsize=2,	                   % sets default tabsize to 2 spaces
	title=\lstname                   % show the filename of files included with \lstinputlisting; also try caption instead of title
}

\definecolor{gray}{rgb}{0.4,0.4,0.4}
\definecolor{darkblue}{rgb}{0.0,0.0,0.6}
\definecolor{cyan}{rgb}{0.0,0.6,0.6}

\definecolor{javared}{rgb}{0.6,0,0} % for strings
\definecolor{javagreen}{rgb}{0.25,0.5,0.35} % comments
\definecolor{javapurple}{rgb}{0.5,0,0.35} % keywords
\definecolor{javadocblue}{rgb}{0.25,0.35,0.75} % javadoc


\lstdefinelanguage{FLY} {
  morestring=[b]",
  morecomment=[l]{//},
  morecomment=[s]{/}{/},
  identifierstyle=\color{black},
  keywordstyle=\color{javapurple},
  morekeywords={func,var,while, if, then, else,for, in, println, fly ,on,thenall,as,then,require,native}% list your attributes here
}

\lstset {
    language=FLY,
    basicstyle=\ttfamily\scriptsize,    
    keywordstyle=\color{javapurple}\bfseries,
    stringstyle=\color{javared},
    commentstyle=\color{javagreen},
    morecomment=[s][\color{javadocblue}]{/*}{/},
    numbers=left,
    numberstyle=\tiny\color{black},
    stepnumber=1,
    numbersep=10pt,
    tabsize=4,
    showspaces=false,
    showstringspaces=false
}


\begin{document}
    \begin{titlepage}
        \begin{center}
            \epsfig{file=figures/logo_standard.jpg,width=2.5truecm}\\[0.2truecm]
            {\Large Universit\`a degli Studi di Salerno}\\[0.2truecm]
            {\large Dipartimento di Informatica}\\
            \hrulefill
            \vfill
            {\large Tesi di Laurea di I livello in }\\[0.2truecm]
            {\Large Informatica}\\
            \vfill\vfill
            {\Huge Adversarial Attacks on Vision-based Deep Neural Networks in Autonomous Driving Vehicles}
            \vfill\vfill
            
            
            {\bf Relatore} \hfill {\bf Candidato}\ \ \\
            Giuseppe Scanniello \hfill Michelangelo Esposito\\
            {\bf Correlatore} \hfill {\bf }\ \ \\
            Dott. Nome Cognome \hfill \ \ \\
            
            \vfill
            \hrulefill 
            
            Academic Year 2021-2022
        
        \end{center}
    \end{titlepage}


    \pagenumbering{roman}
    \chapter*{Abstract}
    This thesis investigates the applicability of Test-Driven Development (TDD) for embedded systems development. Embedded systems are computer systems that perform a specific function and are integrated into larger products; these systems have unique characteristics, such as real-time constraints and resource limitations, which can make traditional software development and testing approaches difficult to apply. TDD is a software development methodology that emphasizes writing automated tests before writing code: the research in this thesis evaluates the feasibility and effectiveness of using TDD for embedded systems development through a case study involving .... 
The results of the study indicate that TDD can be successfully applied to embedded systems development, and can lead to more robust and maintainable code. The thesis also provides insights into the challenges that must be overcome when using TDD for embedded systems development and suggests best practices for addressing those challenges.
    
    \tableofcontents
    \pagestyle{plain}

    \input{sections/list_of_figures.tex}

    \chapter*{List of Acronyms and Abbreviations}
    \begin{acronym}
    \acro{GA}{Genetic Algorithm}
\end{acronym}

\begin{acronym}
    \acro{AI}{Artificial Intelligence}
\end{acronym}

\begin{acronym}
    \acro{OO}{Object Oriented}
\end{acronym}

\begin{acronym}
    \acro{WS}{Whole Suite}
\end{acronym}

\begin{acronym}
    \acro{SUD}{System Under Design}
\end{acronym}

\begin{acronym}
    \acro{SUT}{System Under Test}
\end{acronym}

\begin{acronym}
    \acro{JNI}{Java Native Interface}
\end{acronym}


%%% algorithms %%%
\begin{acronym}
    \acro{NSGA-II}{Non-dominated Sorting Genetic Algorithm-II}
\end{acronym}

\begin{acronym}
    \acro{SPEA2}{Strength Pareto Evolutionary Algorithm}
\end{acronym}

\begin{acronym}
    \acro{MOSA}{Multi-Objective Sorting Algorithm}
\end{acronym}

\begin{acronym}
    \acro{DynaMosa}{Dynamic Multi-Objective Sorting Algorithm}
\end{acronym}

\begin{acronym}
    \acro{LIPS}{}
\end{acronym}


%%% IoT %%%
\begin{acronym}
    \acro{IoT}{Internet of Things}
\end{acronym}

\begin{acronym}
    \acro{ES}{Embedded System}
\end{acronym}

\begin{acronym}
    \acro{ CPS}{CyberPhysical System}
\end{acronym}

    \chapter{Introduction}
    \setcounter{page}{1} 	% must only follow the first chapter
    \pagenumbering{arabic}	% must only follow the first chapter
    Things to add:
\begin{itemize}
    \item where different types of coverage are useful
    \item Add formulas for coverage criteria
\end{itemize}

    \chapter{Problem formulation}
    \newpage
\section{An overview on search problems}
As humans, everyday we perform search: from looking for our car keys in the house to searching for a new book to purchase, this action is engraved in our daily life.
In the same way, search is one of the most fundamental operations on which computer science has focused since its early days: performing efficient search is the core operation of many classes of algorithms, such as the ones responsible for route planning, computer vision, robotics, automated software testing, puzzle solving, and many others.


Any search problem is typically defined by: 
\begin{itemize}
    \item A search space: the set of elements in which we search for our solution. Examples include paths in a graph, the numbers to insert in a sorted list, or the web pages accesses by a web index.
    \item A condition that defines the characteristics of candidate solutions.
\end{itemize}

\textbf{formal problem definition with graphs, distance functions, ...}

A search algorithm will therefore examine the search space according to some criteria and attempt to find a suitable solution. Finding any candidate solution is just one of the objective of search problems however, often times we are interested in finding the best possible solution, the optimum; such problems are referred to as optimization problems.

The exploration of the search space can happen in many different ways. The most basic approach consists of exploring the elements one by one, till a solution/the optimum is found, in a brute-force manner. Applying this simple solution for problems whose search space is somewhat limited can be done without too many repercussions on execution time, however, given the exponential size of the search spaces of most practical problems, a brute-force approach is infeasible. 

For problems in which we have no choice but to employ brute-force search, there may be some room for improvement by using heuristics. Heuristics are estimates of the distance function to reach a goal state from the current node \cite{HeuristicSearch}.


Therefore, what it is preferred to obtain a solution that may not be the optimum, but rather it is "good enough" for our objective. This approach is called local search.

 
\textbf{Hill climbing, simulated annealing, GAs...}
Genetic Algorithms (GAs) are an example of an
evolutionary search approach for test case generation; starting from an initial, often randomly generated, population of 
test cases, the algorithm keeps evolving the individuals according to simulated natural evolution theory principles.
In this context, a typical fitness function of a GA would measure the distance between the execution trace of the generated test cases
and the coverage targets.





\newpage
\section{Driving simulators for self-driving vehicles}

\begin{itemize}
    \item BeamNG
    \item CARLA \cite{Dosovitskiy17} is another open source simulator built specifically for self-driving vehicles in mind. Similarly to BeamNG, it features an API written in Python to actively interact with the simulation any manipulate traffic, pedestrians, weather conditions and more parameters.
    Furthermore, an interesting feature of CARLA is its runtime texture streaming; this features allows developers to change the texture of every object in the scene during runtime. Such a feature can be particularly useful to tune a driving model to continuously deal with adversarial attacks.
    Finally, CARLA simulations can run without rendering any asset, thus allowing runs to be executed much quicker.
    \item Mathematical models: 

\end{itemize}


\newpage
\section{Adversarial attacks}

Most vision-based recognition software on ADS is based on CNNS; often, CNN-based deep learning models are vulnerable to the so called adversarial, 

There can be small, pixel-level changes to an image that will cause the AI model to incorrectly interpret it or, on the other hand, a completely new image can be used to trick to model into thinking it is something else. The first kind is particularly dangerous, since such changes can be invisible to the human eye, and thus harder to detect.

There are mainly two categories of methods to achieve adversarial attacks, namely, optimization-based methods and fast gradient step method (FGSM)-based approach.


study on street signs: K. Eykholt, I. Evtimov, E. Fernandes, B. Li, A. Rahmati, C. Xiao,
A. Prakash, T. Kohno, and D. Song, “Robust Physical-World Attacks on
Deep Learning Visual Classification,” in 2018 IEEE/CVF Conference on
Computer Vision and Pattern Recognition. IEEE, 2018, pp. 1625–1634.


In general, adversarial attacks are organized in three categories: evasion, poisoning, and extraction attacks:
\begin{itemize}
    \item Evasion attacks modify the input to a classifier such that it is misclassified, while keeping the modification as small as possible. Evasion attacks can be black-box or white-box: in the white-box case, the attacker has full access to the architecture and parameters of the classifier. For a black-box attack, clearly this is not the case.
    \item In poisoning attacks, attackers have the opportunity of manipulating the training data to significantly decrease the overall performance, cause targeted misclassification or bad behavior, and insert backdoors and neural trojans
    \item Extraction attacks aim to develop a new model, starting from a proprietary black-box model, that emulate the behavior of the original model.
\end{itemize}


    \chapter{Literature}
    In this chapter we provide a general overview of the literature concerning 3D object detection in self-driving vehicles, as well as the solutions employed to deal with adversarial attacks on their deep learning models

\section{Adversarial attacks on LiDAR data}
Some studies were recently performed on the safety of LiDAR-based 3D object detection models for self-driving vehicles \cite{DBLP:journals/corr/abs-1907-05418} and \cite{DBLP:conf/cvpr/TuRMLYDCU20}.


These works however only considered the LiDAR modality, while most AVs employ also stereo images captured by RGB cameras.


Example algorithm
\begin{algorithm}[H]
    \caption{MOSA}

    \SetKwInOut{Input}{input}
    \SetKwInOut{Output}{output}
    \SetKwBlock{Beginn}{beginn}{ende}

    \DontPrintSemicolon   

    \Input {
        $ U = \{u_1,...,u_m\} $ the set of coverage targets of a program \newline 
        Population size $ M $
    }
    \Output {
        A test suite $ T $ \newline
    }
    \Begin {
        $ t \gets 0 $\;
        $ P_t \gets $ RANDOM-POPULATION($ M $)\;
        $ archive \gets $ UPDATE-ARCHIVE($ P_t $, \O)\;

        \While{$ not (search\_budget\_consumed) $}{
            $ Q_t \gets $ GENERATE-OFFSPRING($ P_t $)\;
            $ archive \gets $ UPDATE-ARCHIVE($ Q_t $, $ archive $)\;
            $ R_t \gets P_t \cup Q_t $\;
            $ F \gets $ PREFERENCE-SORTING($ R_t $)\;
            $ P_{t + 1} \gets 0 $\;
            $ d \gets 0 $\;

            \While{$ (|P_{t + 1}| + |F_d| \leqslant M) $}{
                CROWDING-DISTANCE-ASSIGNMENT($ F_d $)
                $ P_{t + 1} \gets P_{t + 1} \cup F_d $\;
                $ d \gets d + 1 $\;
            }

            Sort($ F_d $) // according to the crowding distance\;
            $ P_{t + 1} \gets P_{t + 1} \cup F_d[1: (M - |P_{t + 1}|)] $\;
            $ t \gets t + 1 $\;
        }
        $ T \gets archive $\;
    }
\end{algorithm}


    \chapter{Conclusions}
    \input{sections/conclusions.tex}

    \nocite{*}
    \printbibliography[title={Bibliography}] 

\end{document}
